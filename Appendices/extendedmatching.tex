\chapter{Extended Pattern Matching Instantiation}
\label{sec:extendedmatching}
This appendix contains notes on instantiating holes in a match expression according also to the structure of the patterns in a match expression.

To implement this, the scrutinee needs to be matched against a pattern \textit{and} instantiated at the same time. Crucially, instead of just giving up during indeterminate matches, the indeterminate part can be instantiated until it matches.

However, this is not always enough to actually \textit{allow} destructuring using that branch in a match statement. The possibility that \textit{more specific} patterns could be present above the current branch means the resulting instantiation might still be result in an indeterminate match. For example, the following would be an indeterminate match:

\begin{figure}[H]
\texttt{case ?::? | [] => [] | x::y::[] => [] | x::xs => xs}

\caption{More Specific Matches}
\end{figure}

To account for this, we can take ideas from pattern matrix techniques for producing exhaustivity warnings, \cite{PatternMatchingWarnings}. That is, we could generate a set of patterns which explicitly do not match any of the previous branches, then intersect those patterns with the current branch, and instantiate according to this intersection.