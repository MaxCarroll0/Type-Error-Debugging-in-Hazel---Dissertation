\chapter{Associativity of Bounded DFS bind}
\label{sec:BDFSbindassoc}
\begin{figure}
\begin{minted}{reason}
((m >>= f) >>= g)(bound)
= m(bound) 
  |> map((x, rem_bound) => f(x, rem_bound))
  |> concat
  |> map((y, rem_bound2) => g(y, rem_bound2))
  |> concat
\end{minted}
Vs.
\begin{minted}{reason}
(m >>= (x => f(x) >>= g))(bound)
= m(bound)
  |> map((x, rem_bound) => rem_bound |> (f(x) >>= g))
  |> concat
= m(bound)
  |> map((x, rem_bound) => rem_bound |> (f(x) 
  |> map((y, rem_bound2) => g(y, rem_bound2)) 
  |> concat))
  |> concat
= m(bound)
  |> map((x, rem_bound) => f(x, rem_bound) 
  |> map((y, rem_bound2) => g(y, rem_bound2))
  |> concat)
  |> concat
= m(bound)
  |> map((x, rem_bound) => f(x, rem_bound)) 
  |> concat
  |> map((y, rem_bound2) => g(y, rem_bound2))
  |> concat
\end{minted}
\caption{Associativity of Iterative DFS \code{bind}}
\end{figure}
