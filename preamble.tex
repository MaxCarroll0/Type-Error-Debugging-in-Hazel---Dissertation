\usepackage{graphicx}
\usepackage[utf8]{inputenc}
\usepackage[backend=biber, style=numeric, sorting=ydnt]{biblatex}
\usepackage{amsmath}
\usepackage{semantic}
\usepackage{minted}
\usemintedstyle{colorful}
\usepackage{stmaryrd}
\usepackage{wasysym}
\DeclareMathAlphabet{\mathdcal}{U}{dutchcal}{m}{n}
\usepackage{amssymb}

\usepackage[font=small,labelfont=bf]{caption}

\usepackage[dvipsnames]{xcolor}
\usepackage[normalem]{ulem}
\makeatletter
\def\uwave{\bgroup \markoverwith{\lower3.5\p@\hbox{\sixly \textcolor{red}{\char58}}}\ULon}
\font\sixly=lasy6 % does not re-load if already loaded, so no memory problem.
\makeatother

\usepackage{soul}


\newcommand{\hlc}[2][yellow]{{%
    \colorlet{foo}{#1}%
    \sethlcolor{foo}\hl{#2}}%
}
\newcommand{\error}[1]{\hlc[pink!50]{#1}}
\newcommand{\hlcmaths}[2][yellow!50]{\colorbox{#1}{$\displaystyle #2$}}

\newcommand{\code}[1]{\text{\mintinline[escapeinside=||]{reason}{#1}}}
\newcommand{\dyn}{{\color{teal}\texttt{?}}}
\newcommand{\casthl}[4][yellow!50]{\ensuremath{\code{#4}^{\text{\hlc[#1]{${\langle$\code{#2}$ \Rightarrow $\code{#3}$\rangle}$}}}}}
\newcommand{\cast}[3]{\casthl[red!0]{#1}{#2}{#3}}
\newcommand{\casterror}[3]{\casthl[pink!50]{#1}{#2}{#3}}

\newcommand{\hole}[1][]{{\color{teal}\llparenthesis} #1 {\color{teal}\rrparenthesis}}
\newcommand{\gap}{{\color{teal}\Box}}
\newcommand{\cmark}{{\color{blue}\Circle}}
\newcommand{\scast}[2]{{\color{blue}\langle} #1 {\color{blue}\Rightarrow} #2 {\color{blue}\rangle}}
\newcommand{\scasterror}[2]{{\color{red}\langle} #1 {\color{red}\Rightarrow} \dyn {\color{red}\nRightarrow} #2 {\color{red}\rangle}}
\newcommand{\scastcast}[3]{{\color{blue}\langle} #1 {\color{blue}\Rightarrow} #2 {\color{blue}\Rightarrow} #3 {\color{blue}\rangle}}

\newcommand{\synthesis}[3][\Gamma]{#1 \vdash #2 {\color{BrickRed}\ \Rightarrow}\ #3}
\newcommand{\synthesissliceindexed}[3][\Gamma]{#1 \vdash #2 \dashv_{\color{BrickRed}\Rightarrow}#3}
\newcommand{\synthesisslice}[4][\Gamma]{#1 \vdash #2 {\color{BrickRed}\ \Rightarrow}\ #3 \dashv #4}

\newcommand{\analysis}[3][\Gamma]{#1 \vdash #2 {\color{BlueViolet}\ \Leftarrow}\ #3}
\newcommand{\typeassignment}[3][\Delta;\Gamma]{#1 \vdash #2 : #3}
\newcommand{\analysisslice}[5][\Gamma]{#1 \vdash #2\{#3\} {\color{BlueViolet}\ \Leftarrow}\ #4 \dashv #5}

%\newcommand{\typeassignment}[3][\Delta;\Gamma]{#1 \vdash #2 : #3}
\newcommand{\elaborationSynthesis}[5][\Gamma]{#1 \vdash #2 {\color{BrickRed}\ \Rightarrow\ } #3 \leadsto #4 \dashv #5}
\newcommand{\elaborationAnalysis}[6][\Gamma]{#1 \vdash #2 {\color{BlueViolet}\ \Leftarrow\ } #3 \leadsto #4 : #5 \dashv #6}

\newcommand{\funmatch}{\blacktriangleright_\rightarrow}
\newcommand{\typematch}{\blacktriangleright_{\text{type}}}
\newcommand{\ground}[1][\tau]{#1\text{ ground}}
\newcommand{\groundmatch}{\blacktriangleright_{\text{ground}}}
\newcommand{\final}[1][d]{#1\text{ final}}
\newcommand{\val}[1][d]{#1\text{ val}}
\newcommand{\boxedval}[1][d]{#1\text{ boxedval}}
\newcommand{\indet}[1][d]{#1\text{ indet}}

\newcommand{\contextualsub}[1][d/u]{\llbracket #1 \rrbracket}

\newcommand{\synthesisSlice}[4][\Gamma]{#1 \vdash #2 {\color{BrickRed}\ \Rightarrow}\ #3\ \|\ #4}
\newcommand{\analysisSlice}[4][\Gamma]{#1 \vdash #2 {\color{BlueViolet}\ \Leftarrow}\ #3\ \|\ #4}

\newcommand{\elaborationSynthesisSlice}[6][\Gamma]{#1 \vdash #2 {\color{BrickRed}\ \Rightarrow\ } #3 \leadsto #4 \dashv #5\ \|\ #6}
\newcommand{\elaborationAnalysisSlice}[7][\Gamma]{#1 \vdash #2 {\color{BlueViolet}\ \Leftarrow\ } #3 \leadsto #4 : #5 \dashv #6\ \|\ #7}

\newcommand{\type}[1]{\llbracket #1 \rrbracket}

\addbibresource{References/refs.bib}
\makeindex


\usepackage{hyperref}
\hypersetup{
    colorlinks=true,
    linkcolor=blue,
    filecolor=magenta,      
    urlcolor=cyan,
    pdftitle={\theauthor: \thetitle},
    pdfpagemode=FullScreen,
    }

\usepackage{cleveref}


\newtheorem{conjecture}{Conjecture}
\newtheorem{definition}{Definition}

\urlstyle{same}
\newcommand{\secref}[1]{\S\ref{#1}}

%TC:group tabular 1 1
%TC:group table 0 1
%TC:macro \footnote [text]

\immediate\write18{texcount -inc -sum -0 \jobname.tex > wordcount.aux }
\newcommand{\wordcount}{\input{wordcount.aux}}

\setsecnumdepth{subsection}