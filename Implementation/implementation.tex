\chapter{Implementation}\label{chap:Implementation}
This project was conducted in two major phases:

First, I constructed a core mathematical theory for \textit{type slicing} and \textit{cast slicing} formalising what these ideas actually were and considered the changes to the system presented by Seidel et al. for the \textit{type error witnesses search procedure} to work in Hazel.  

Then, I implemented the theories, making it suitable for implementation and extending it to the majority of the Hazel language. Further, suitable deviations from the theory were made upon critical evaluation and are detailed throughout.

\textbf{Annotate the above with the relevant section links!}
\section{Type Slicing Theory}\label{sec:TypeSlicingTheory}
I develop \textit{type slicing} as a mechanism to aid programmers in understanding \textit{why} a term has a given type via static means. Three slicing mechanism have been devised with differing characteristics, all of which associate terms with their typing derivation to produce a \textit{program slice}. 

The first two criteria attempt to give insight on the structure of the typing derivations, and hence how types are decided. While the third criterion gives a complete picture of the regions of code which, if changed, could cause a change in type of the whole expression.

\subsection{Program Slices}
A \textit{program slice} $\rho$, is a pair $(\varsigma \mid \gamma)$, consisting of an \textit{expression slice} and \textit{typing context slice} respectively which are calculated based on some typing \textit{criterion}\footnote{One of the three slicing mechanisms.} based on the typability of the slice $\varsigma$ under context $\gamma$. 

Formally, an expression slice is a Hazel external expression which may have sub-terms \textit{omitted}. I represent omissions from expressions by replacing the respective sub-term with a \textit{gap}, notated $\gap$. This is the same definition of a slice as presented in \cite{FunctionalProgExplain}.\footnote{With their `holes' equating with my `gaps'. Different terminology used to distinguish with Hazel's holes} 

This gives the following extended syntax of expression slices, $\varsigma$, extending \cref{fig:syntax}:
\[\varsigma ::= c \mid x \mid \lambda x : \tau.\ \varsigma \mid \lambda x.\ \varsigma \mid \varsigma(\varsigma) \mid \hole^u \mid \hole[\varsigma]^u \mid \varsigma : \tau \mid \gap\]

These slices are then allowed to be \textit{typed} by representing gaps $\gap$ by holes of fresh metavariables $\hole^u$. It turns out that none of the proposed criteria ever highlight (retain) hole terms,\footnote{As holes have the dynamic type.} hence gaps can be directly represented as holes internally without loss of generality. From here-on consider $\gap$ as interchangeable with a hole $\hole^u$ of fresh metavariable $u$.

We then have a \textit{precision} relation on expression slices, $\varsigma_1 \sqsubseteq \varsigma_2$ meaning $\varsigma_1$ is less or equally precise than $\varsigma_2$, that is $\varsigma_1$ matches $\varsigma_2$ structurally except that some subterms may be gaps. For example, see this precision chain:
\[\gap \sqsubseteq\gap + \gap\sqsubseteq 1 + \gap \sqsubseteq 1 + 2\]
We have that $\sqsubseteq$ is a partial order (\textbf{cite}), that is, satisfies relexivity, antisymmetry, and transitivity. Respectively:
\[\inference{}{\varsigma \sqsubseteq \varsigma} \quad \inference{\varsigma_1 \sqsubseteq \varsigma_2 & \varsigma_2 \sqsubseteq \varsigma_1}{\varsigma_1 = \varsigma_2} \quad \inference{\varsigma_1 \sqsubseteq \varsigma_2 & \varsigma_2 \sqsubseteq \varsigma_3}{\varsigma_1 \sqsubseteq \varsigma_3}\]
We also have a \textit{bottom} (least) element, $\gap \sqsubseteq \varsigma$ (for all $\varsigma$). This relation is trivially extended to include complete expressions $e$ which satisfy that: if $e \sqsubseteq \varsigma$ then $e = \varsigma$, i.e. complete terms are always \textit{unique} upper bounds of precision chains.

\textit{Context slices} are simply a typing context $\Gamma$, which is used to represent the notion of \textit{relevant typing assumption}. Typing contexts are just functions mapping variables to types notated $x : \tau$ (see \cref{sec:TypingJudgements}). Functions are sets, so they also have a partial order of subset inclusion, $\subseteq$. Again, we have a bottom element, $\emptyset$.

The precision relation and subset inclusion can be extended pointwise to give a partial order, $\sqsubseteq$, on program slices:
\[(\varsigma_1\mid \gamma_1) \sqsubseteq (\varsigma_2\mid \gamma_2) \iff  \varsigma_1 \sqsubseteq \varsigma_2 \text{ and } \gamma_1 \subseteq \gamma_2\]

Program slices will often be grouped and indexed upon expressions and typing contexts, $P_e[\Gamma]$ which contains all slices $\rho \sqsubseteq (e, \Gamma)$. So, the set $P_e[\Gamma]$ forms a lattice (\textbf{cite}) with unique least upper bound $(e, \Gamma)$ and greatest lower bound $(\gap, \emptyset)$.

Such 

\subsection{Criterion 1: Synthesis Slices}
\label{sec:SynthesisSlices}
For \textit{synthesis type slices} we consider an expression synthesising a type $\tau$ under some context $\Gamma$:
\[\synthesis{e}{\tau}\]
And consider the slices in $P_e[\Gamma]$ and attempt to find the minimum slice $\rho = (\varsigma\mid \gamma)$ constraining that $\rho$ also synthesises the same type $\tau$ under the restricted context $\gamma$:
\[\synthesis[\gamma]{\varsigma}{\tau}\]
Where minimality requires that no other (strictly) less precise slice satisfies the criterion. That is: for any slice $\rho' = (\varsigma'\mid \gamma')$, if $\synthesis[\gamma']{\varsigma'}{\tau}$ and $\rho' \sqsubseteq \rho$, then $\rho' = \rho$.

I conjecture that, under the Hazel type system, there exists a unique minimum slice for each $\synthesis{e}{\tau}$:\footnote{Would follow from uniqueness of typing derivations in Hazel.}
\begin{conjecture}[Uniqueness]\label{conj:SynthesisSliceUniqueness}
If $\rho$ and $\rho'$ are \textit{minimum synthesis slices} for $\synthesis{e}{\tau}$, then $\rho = \rho'$.
\end{conjecture}

These slices can be found by omitting portions of the program which are \textit{type checked}. If, $\analysis{e}{\tau}$, then by use of the subsumption rule we also have that $\analysis{\gap}{\tau}$:
\[\inference[Subsumption]{\synthesis{\gap}{\dyn} & \tau \sim \dyn}{\analysis{e}{\tau}}\] 
As the dynamic type is consistent with any type: $\dyn \sim \tau$.

Then, to find the \textit{minimum synthesis slice}, we can mimic the Hazel type synthesis rules (see \cref{fig:typing}), replacing uses of type analysis with gaps. Creating a judgement $\synthesisslice{e}{\tau}{\rho}$ meaning: \textit{$e$ that synthesises type $\tau$ under context $\Gamma$ produces minimum synthesis slice $\rho$}.

To demonstrate, the expression slice of a variable $x$ can only be either $x$, requiring the use of $x : \tau$ from the context:
\[
\inference[SVar]{x : \tau \in \Gamma & \tau \neq \dyn}{\synthesisslice{x}{\tau}{[x\mid x:\tau]}}\]
But if $x : \dyn$, then the (empty) slice $[\gap, \emptyset]$ also synthesises $\dyn$, so instead use this. 

For functions, we can recursively find the slice of the function body (which synthesises it's type in the original rules, hence having a minimum synthesis slice) and place inside a function. 
If the assumption $x : \tau_1$ was \textit{required} in synthesising that type, then this name must be present in the expression slice and the context slice no longer requires this assumption to type check the sliced function:
\[\inference[\tiny SFun]{\synthesisslice[\Gamma,x:\tau_1]{e}{\tau_2}{[\varsigma \mid \gamma, x : \tau_1]} }{\synthesisslice{\lambda x:\tau_1.\ e}{\tau_1 \to \tau_2}{[\lambda x : \tau_1.\ \varsigma \mid \gamma]}}\]
Otherwise, if $\gamma$ does not use variable $x$ then this binding may be omitted:
\[\inference[\tiny SFunConst]{\synthesisslice[\Gamma,x:\tau_1]{e}{\tau_2}{[\varsigma \mid \gamma]} & x \not \in \mathrm{dom}(\gamma)}{\synthesisslice{\lambda x:\tau_1.\ e}{\tau_1 \to \tau_2}{[\lambda \gap : \tau_1.\ \varsigma \mid \gamma]}}\]

For function applications we can simply omit the argument, while the slice for the function can be obtained as it synthesises it's type.
\[\inference[\tiny SApp]{\synthesisslice{e_1}{\tau_1}{[\varsigma_1 \mid \gamma_1]}}{\synthesis{e_1(e_2)}{\tau}{[\varsigma_1(\gap) \mid \gamma_1]}}\]
The remaining rules are in \cref{fig:SynthesisSlices}.

It is \textit{expected} \textbf{(Proof TODO)} that these rules do indeed always produce a slice for any expression which synthesises a type, and that this slice is minimal:
\begin{conjecture}[Correctness]
\label{conj:SynthesisSliceUniqueness}
If $\synthesis{e}{\tau}$ then:
\begin{itemize}
\item $\synthesisslice{e}{\tau}{\rho}$ where $\rho = [\varsigma \mid \gamma]$ with $\synthesis[\gamma]{\varsigma}{\tau}$.
\item For any $\rho' = [\varsigma' \mid \gamma'] \sqsubseteq [e\mid \Gamma]$ such that $\synthesis[\gamma']{\varsigma'}{\tau}$ then $\rho \sqsubseteq \rho'$.
\end{itemize}
\end{conjecture}
\subsection{Criterion 2: Analysis Slices}\label{sec:AnalysisSlices}
Any type analysed against must come from a type synthesis, so will be paired with an expression in whatever context it was in. However, this is now no longer just a property about the term we are looking at, but instead the \textit{whole} context: treat this context as a new input, making a new hypothetical judgement. 

See written notes for detailed ideas.


\subsection{Criterion 3: \textit{good\_name\_here}}


\subsection{Join Types}
These are the only way a \textit{new type} is created other than synthesis, add some machinery to demonstrate how a program slice could be constructed by taking the `deepest' branch in the join and working out if it was the left/right branch etc.

\section{Cast Slicing Theory}\label{sec:CastSlicingTheory}

Fairly trivial, just treat slices as types and decompose accordingly. The whole reason of indexing by type was to allow this.

The idea of it being a minimal program slice producing the same cast doesn't really work here due to dynamics. Explore the maths of this. Either way, it is a useful construct in practice. Exploring this in more detail, looking at \textit{dynamic program slicing} could be a good future direction.


\subsection{Indexing Program Slices by Types}


\subsection{Elaboration}

\subsection{Dynamics}

\subsection{Cast Dependence}
This in combination with the indexed slices could have some nice mathematical properties.

Though, these would need to retrieve information about parts of slices that were lost when decomposing slices. i.e. when a slice $\tau_1 \to \tau_2$ extracts the argument type $\tau_2$, the slicing criterions lose track of the original lambda binding. A way to reinsert these bindings such that we get a minimal term which \textit{Evaluates to the same cast} e.g. But this will have lots of technicalities with correctly tracking the restricted contexts $\gamma'$ for closures (i.e. functions returning functions which have been applied once). \textbf{TALK ABOUT THIS IN THE FURTHER DIRECTIONS SECTION.}

\section{Proofs}\label{sec:Proofs}
Do if there is time. Prove that analysis slice contexts actually make sense, that they maintain a valid term that is \textit{still} minimal.

\section{Type Slicing Implementation}\label{sec:TypeSlicingImplementation}
\subsection{Type Slice Data-Type}\label{sec:TypeSliceDataType}
Detail initial implementation (just tagging existing types). Compare with final implementation, quantitative numbers for improve could be obtained but would require quite a bit of coding work... Polymorphic Variants :))
\subsubsection{Code Slices}\label{sec:CodeSlices}
id based. Has ctx used but not actually required as I decide to directly store typslices in context (explain how this differs from the theory).

Mention that full slices as in the theory is very inefficient.
\subsubsection{Integration with Existing Type Data-Type}
Use of `Typ. Explain how this allows it to easily be disabled and saves space (maybe).
\subsubsection{Synthesis \& Analysis Slices}
Incremental/Global. Detail the choice to not annotate many analysis slices.
\subsubsection{Mapping Functions}
\subsubsection{Type Slice Joins}
Just unions of the lists. Double check that we can't have elements from more than one branch highlighted.\footnote{Type variables might make this possible??}

\subsection{Static Type Checking}\label{sec:TypeChecking}
Detail the statics and Info types. Explain the distinction between Self.re and Mode.re, which links very nicely with the synthesis and analysis slice distinctions.

\subsection{Elaboration}\label{sec:Elaboration}
Make casts use type slices, insertion is mostly the same. Just need to ensure that the slices are preserved (i.e. types are threaded through without being replaced)

\subsection{User Interface}
Click on analysis or synthesis slices from context inspector

\section{Cast Slicing Implementation}\label{sec:CastSlicingImplementation}
\subsection{Cast Transitions}

Cast transition and unboxing logic

\subsection{User Interface}
Mainly talk about the Model-view architecture and passing the cursor into the evaluator view to allow 


\section{EV\_MODE Evaluation Abstraction}
Very complex!!!

\section{Indeterminate Evaluation}\label{sec:IndetEval}
\subsection{Futures Data-Type}\label{sec:Futures}
Lazy lists, often infinite
\subsection{Hole Instantiation}\label{sec:HoleInstantiation}
Small Hole hypothesis, quick check


\subsubsection{Choosing which Hole to Instantiate}
Use EV\_Mode to select next hole to instantiate

\subsubsection{Synthesising Terms for Types}
Difficulties instantiating strings...

\subsubsection{Substituting Holes}\label{sec:HoleSubstitutionImplementation}
Detail that this was an unexpected extra task, and is therefore not exactly the same as hole substitution as detailed in Preparation (i.e. no metavars or contexts annotated on holes, but it is enough for the search procedure to work)

\subsection{Cast Laziness}\label{sec:CastLaziness}
Ref the original cast slicing paper, which is not lazy apparently? Laziness sort of breaks the idea that runtime errors evaluate to cast errors. There can be compound values of the wrong type being cast, but the error will only be found upon accessing parts of the compound type.

Making casts eager is a major change to the actual transitions.

Eager casts also catch `spurious' errors (see Evaluation).


\subsection{User Interface}

\section{Evaluation Stepper}\label{sec:Stepper}
\subsection{Evaluation Contexts}
\subsection{Customisable Hole Instantiation}
\subsection{User Interface}


\section{Search Procedure}\label{sec:SearchProcedure}
\subsection{Detecting Relevant Cast Errors}
i.e. failed cast at head of term
\subsection{Filtering Indeterminate Evaluation}
Done via EV\_MODE similarly to finding which hole to instantiate.

\subsection{Iterative Deepening}\label{sec:IterativeDeepening}
Required after evaluating that infinite loops break the thing
\subsection{User Interface}


