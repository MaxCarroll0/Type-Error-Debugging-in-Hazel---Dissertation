\chapter{Implementation}\label{chap:Implementation}
This project was conducted in two major phases:

First, I constructed a core mathematical theory for \textit{type slicing} and \textit{cast slicing} formalising what these ideas actually were and considered the changes to the system presented by Seidel et al. for the \textit{type error witnesses search procedure} to work in Hazel.  

Then, I implemented the theories, making it suitable for implementation and extending it to the majority of the Hazel language. Further, suitable deviations from the theory were made upon critical evaluation and are detailed throughout.

\textbf{Annotate the above with the relevant section links!}
\section{Type Slicing Theory}\label{sec:TypeSlicingTheory}
\textbf{Replace all occurrences of `typing context' with `typing assumptions' to avoid name clash with expression/program contexts.}

I develop \textit{type slicing} as a mechanism to aid programmers in understanding \textit{why} a term has a given type via static means. Three slicing mechanism have been devised with differing characteristics, all of which associate terms with their typing derivation to produce a \textit{expression typing slice}. 

The first two criteria attempt to give insight on the structure of the typing derivations, and hence how types are decided. While the third criterion gives a complete picture of the regions of code which, if changed, could cause a change in type of the whole expression.

\textbf{Make some brief arguments into why the first two criteria are still useful.}

\textbf{PLACE ALL IMPORTANT DEFINITIONS INSIDE DEFINITION ENVIRONMENTS}

\subsection{Expression Typing slices}
A \textit{expression typing slice} $\rho$, is a pair $\varsigma^\gamma$, consisting of an \textit{expression slice} $\varsigma$ and \textit{typing context slice} $\gamma$ which are calculated based on some typing \textit{criterion}\footnote{One of the three slicing mechanisms.} based on the typability of the slice $\varsigma$ under context $\gamma$. 

Intuitively, an expression slice is a Hazel external expression highlighting the sub-terms of relevance to the \textit{typing criterion}. For example if my criterion is to \textit{omit terms which are typed as} \code{Int}, then the following expressions highlights as:

\[\hlcmaths[yellow!30]{(\lambda x: \code{Int}.\ \lambda y : \code{Bool}.}\ x\hlcmaths[yellow!30]{)(}1\hlcmaths[yellow!30]{)}\]

Formally, I represent this by specifying which sub-terms are omitted in the highlighted expression. So, Replace each omitted sub-term with a \textit{gap}, notated $\gap$. This is the same definition of a slice as presented in \cite{FunctionalProgExplain}.\footnote{With their `holes' equating with my `gaps'. Different terminology used to distinguish with Hazel's holes} i.e. representing the above highlighting we get slice:
\[(\lambda x : \code{Int}.\ \lambda y : \code{Bool}.\ \gap)(\gap)\]


Additionally, it is useful to omit variable names. For this I introduce \textit{patterns} $p$ for variable bindings: 
\[p ::= \gap \mid x\]

This gives the following extended syntax of expression slices, $\varsigma$, extending \cref{fig:syntax}:
\[\varsigma ::= \gap \mid  c \mid x \mid \lambda p : \upsilon.\ \varsigma \mid \lambda x.\ \varsigma \mid \varsigma(\varsigma) \mid \hole^u \mid \hole[\varsigma]^u \mid \varsigma : \upsilon\]
Where $\upsilon$ are types, similarly with potential omitted sub-term gaps:
\[\upsilon ::= \gap \mid \dyn \mid b \mid \upsilon \to \upsilon\]
These slices are then allowed to be \textit{typed} by representing gaps $\gap$ by holes of fresh metavariables $\hole^u$ in \textit{expressions}, fresh variables in \textit{patterns}, and the dynamic type in \textit{types}, see \textbf{(fig APPENDIX)}. From here-on consider $\gap$ as interchangeable with a hole $\hole^u$ of fresh metavariable $u$ or the dynamic type.

We then have a \textit{precision} relation on expression slices, $\varsigma_1 \sqsubseteq \varsigma_2$ meaning $\varsigma_1$ is less or equally precise than $\varsigma_2$, that is $\varsigma_1$ matches $\varsigma_2$ structurally except that some subterms may be gaps, see \textbf{ref appendix}. For example, see this precision chain:
\[\gap \sqsubseteq\gap + \gap\sqsubseteq 1 + \gap \sqsubseteq 1 + 2\]
We have that $\sqsubseteq$ is a partial order (\textbf{cite}), that is, satisfies relexivity, antisymmetry, and transitivity. Respectively:
\[\inference{}{\varsigma \sqsubseteq \varsigma} \quad \inference{\varsigma_1 \sqsubseteq \varsigma_2 & \varsigma_2 \sqsubseteq \varsigma_1}{\varsigma_1 = \varsigma_2} \quad \inference{\varsigma_1 \sqsubseteq \varsigma_2 & \varsigma_2 \sqsubseteq \varsigma_3}{\varsigma_1 \sqsubseteq \varsigma_3}\]
We also have a \textit{bottom} (least) element, $\gap \sqsubseteq \varsigma$ (for all $\varsigma$). This relation is trivially extended to include complete expressions $e$ which satisfy that: if $e \sqsubseteq \varsigma$ then $e = \varsigma$, i.e. complete terms are always upper bounds of precision chains.

An expression slice $\varsigma$ \textit{of} $e$ is a slice such that $e \sqsubseteq e$.

\textit{Typing context slices} are simply a typing context $\Gamma$, which is used to represent the notion of \textit{relevant typing assumption}. Typing contexts are just functions mapping variables to types notated $x : \tau$ (see \cref{sec:TypingJudgements}). Functions are sets, so they also have a partial order of subset inclusion, $\subseteq$. Again, we have a bottom element, $\emptyset$.  These are notated $\gamma$ and if $\gamma \subseteq \Gamma$ then $\gamma$ is a slice \textit{of} $\Gamma$.

The precision relation and subset inclusion can be extended pointwise to give a partial order, $\sqsubseteq$, on expression typing slices:
\[[\varsigma_1\mid \gamma_1] \sqsubseteq [\varsigma_2\mid \gamma_2] \iff  \varsigma_1 \sqsubseteq \varsigma_2 \text{ and } \gamma_1 \subseteq \gamma_2\]

expression typing slices will often be grouped and indexed upon expressions and typing contexts, $P_e^{\Gamma}$ which contains all slices $\rho \sqsubseteq (e, \Gamma)$. So, the set $P_e^{\Gamma}$ forms a lattice (\textbf{cite}) with unique least upper bound $[e, \Gamma]$ and greatest lower bound $[\gap, \emptyset]$. Similarly, an element $\rho$ in $P_e^{\Gamma}$ can referred to as a expression typing slice \textit{of} $e$ under $\Gamma$.
\subsection{Context Slices}
\textbf{add pattern context. }

\newcommand{\C}{\mathdcal{C}}
Formally, an \textit{expression context} $\mathdcal{C}$ is an expressions with \textit{exactly one} sub-term marked as $\cmark$:\footnote{The two separate syntax definition for application allow a \textit{mark} to be in either the left or right expression, but \textit{not both}.}
\[\C ::=  \cmark \mid \lambda x : \tau.\ \C \mid \lambda x.\ \C \mid \C(e) \mid e(\C) \mid \C : \tau\]

Where $\C\{e\}$ substitutes expression $e$ for the mark $\cmark$ in $\C$, the result of this is necessarily an expression. Additionally, contexts are composable: substituting a context into a context, $\C_1\{\C_2\}$ produces another valid context, notate this by $\C_1 \circ \C_2$\footnote{Context can alternatively be though of as functions from expressions to expressions.}.


\newcommand{\Cs}{\mathdcal{c}}
\newcommand{\p}{\mathdcal{p}}
Similarly to expressions, contexts can be extended to \textit{context slices} by allowing slices within:
\[\Cs ::= \cmark \mid \lambda p : \upsilon. \Cs \mid \Cs(\varsigma) \mid \varsigma(\Cs) \mid \Cs : \upsilon\]

However, the precision relation $\sqsubseteq$ is defined differently, requiring that the mark $\cmark$ must remain in the same position in the context structurally speaking. For example $\cmark(\gap) \sqsubseteq \cmark(1)$, but $\cmark \not \sqsubseteq \cmark(1)$. This can be concisely defined by \textit{extensionality} (\textbf{cite}):

\begin{definition}[Context Precision]\label{def:ContextPrecision}
If $\Cs'$ and $\Cs$ are context slices, then $\Cs' \sqsubseteq \Cs$ if and only if, for all expressions $e$, that $\Cs'\{e\} \sqsubseteq \Cs\{e\}$.
\end{definition}
Again, we refer to a context slice $\Cs$ of $\C$ as one satisfying that $\Cs' \sqsubseteq \C$.

We also get that filling contexts preserves the precision relations both on expression slices \textit{and} context slices:
\begin{conjecture}[Context Filling Preserves Precision]
For expression slice $\varsigma$ and context slice $\Cs$. Then if we have slices $\varsigma' \sqsubseteq \varsigma$, $\Cs' \sqsubseteq \Cs$ then also $\Cs'\{\varsigma'\} \sqsubseteq \Cs\{\varsigma\}$.
\end{conjecture}
Therefore, context slices $\Cs$ can be though of as monotone function (\textbf{cite})...

Show that composition is also preserved over precision, i.e. it is a functor.

\textit{Make some references to category theory, i.e. category of slices with morphisms being context slices. Or that slices form categories on expression $e$ and contexts are a functor between expression typing slice categories. i.e. contexts are monotonic functions}

\textbf{rewrite}
The accompanying notion of a \textit{typing co-context slice} is retained representing the notion of \textit{irrelevant typing assumptions}, those assumptions which would be used within portions of the context which were omitted. i.e. for a a term placed in the mark, we could type it using the original context minus the ones used only in typing the context.

This pair of context slice and typing assumptions slice is together referred to as a \textit{context slice} and notated $\p^\gamma$, should $\gamma$ be empty it may be omitted shorthand.

Extend composition to program contexts and substitution of expression typing slices. Again as slices are a superset of expressions, then this extends to expression etc.

When 

\subsection{Type-Indexed Slices}
\textbf{Have context i.e. contexts with a $_$ and let the syntax write it directly as a \textit{type-indexed context slice}, i.e. a type-indexed slice, but with the slice $\rho$s replaced with $\p$s. Then allow easy syntax to create from a type indexed slice i.e.}
\[(\p\{\dyn \mid \rho\})(\cmark) = \p\{\dyn \mid \rho(\cmark)\}\]
Reverse composition order for this?? Also, syntax for going from indexed context to indexed expression i.e. $\{\}$


For \textit{criteria 3 and cast slicing}, there is a need to decompose slices to find sub-slices which contribute to specific portions of a compound type. For example, which part of the expression typing slice was related to the argument type of a function specifically.

\textbf{Give EXAMPLE}

A \textit{type(-indexed) slice} $s$ consists of: a expression typing slice, a context slice, and a \textit{type} \textit{index} $i$. This index is either an \textit{atomic} type label or is \textit{compound}, consisting of type slices conforming to the structure of types:
\[i ::= \dyn \mid b \mid s \to s\]
\[s ::= \p\{i \mid \rho\}\]
The type that a type slice $s$ \textit{represents} is the slice retaining only it's type labels. This will be notated by $\type{s}$, defined inductively:
\[\type{\p\{\dyn\mid \rho\}} = \dyn \quad \type{\p\{b \mid \rho\}} = b\]
\[\type{\p\{s_1 \to s_2 \mid \rho\}} = \type{s_1} \to \type{s_2}\]

This same notation will also be used to extract the type of a type index $i$, similarly defined.

A term $e$ in some context $\C$ will be associated with a \textit{type slice} with the meaning that $\rho$ contains a expression typing slice for typing $e$ and $\p$ contains a context slice for typing $e$ in context $\C$ according to some criterion. Typically the context slice would correspond to type analysis and the expression typing slice would correspond to type synthesis.

The compound type slices must satisfy the crucial property that the sub-terms are sub-\textit{slices} of $\rho$. That is:
\[\p\{\p_1\{i_1 \mid \rho_1\} \to \p_2\{i_2 \mid \rho_2\} \mid \rho\} \implies \p_1\{\rho_1\} \sqsubseteq \rho\text{ and }\p_2\{\rho_2\} \sqsubseteq \rho\]

The precision relation can be extended to slices pointwise upon the expression typing slice and context slice for atomic types $a$ (i.e. $a ::= \dyn \mid b$):
\[\p'\{a \mid \rho'\} \sqsubseteq \p\{a \mid \rho\} \iff \p' \sqsubseteq \p\text{ and } \rho' \sqsubseteq \rho\]
And recursively for compound slices:\footnote{Note, function arguments are \textit{covariant} for this.}
\begin{align*}
\p'\{s_1' \to s_2' \mid \rho'\} \sqsubseteq \p\{s_1 \to s_2 \mid \rho\} \iff &\p' \sqsubseteq \p,\ \rho' \sqsubseteq \rho,\\
&s_1' \sqsubseteq s_1,\text{ and }s_2' \sqsubseteq s_2
\end{align*}

Composition of expression typing slices to the outer context is possible, $(\p' \circ \p)\{i \mid \rho\}$, and is notated shorthand as $\p'\{s\}$ for $s = \p\{i \mid \rho\}$.

Additionally, if $\p$ is empty, $\cmark^\emptyset$, a type slice may be notated without it: $i \mid \rho$. Equally, when $\rho$ is empty, $\gap^\emptyset$, then notate $\p'\{i\}$. This means that if both $\p, \rho$ are both empty then we have $i$ which is syntactically identical to types $\tau$, so we can trivially treat types as \textit{empty type slices}.  

Then function matching $\funmatch$ can be extended to slices in multiple different ways with different uses depending on the context, see the appendix \textbf{ref}.

\subsection{Criterion 1: Synthesis Slices}
\label{sec:SynthesisSlices}
For \textit{synthesis type slices} we consider an expression synthesising a type $\tau$ under some context $\Gamma$:
\[\synthesis{e}{\tau}\]
And consider the slices in $P_e^{\Gamma}$ and attempt to find the minimum slice $\rho = [\varsigma\mid \gamma]$ constraining that $\rho$ also synthesises the same type $\tau$ under the restricted context $\gamma$:
\[\synthesis[\gamma]{\varsigma}{\tau}\]
Where minimality requires that no other (strictly) less precise slice satisfies the criterion. That is: for any slice $\rho' = [\varsigma'\mid \gamma']$, if $\synthesis[\gamma']{\varsigma'}{\tau}$ and $\rho' \sqsubseteq \rho$, then $\rho' = \rho$.

\textbf{GIVE CONCRETE EXAMPLE HERE, use highlighting}

I conjecture that, under the Hazel type system, there exists a unique minimum slice for each $\synthesis{e}{\tau}$:\footnote{Would follow from uniqueness of typing derivations in Hazel.}
\begin{conjecture}[Uniqueness]\label{conj:SynthesisSliceUniqueness}
If $\rho$ and $\rho'$ are \textit{minimum synthesis slices} for $\synthesis{e}{\tau}$, then $\rho = \rho'$.
\end{conjecture}

These slices can be found by omitting portions of the program which are \textit{type checked}. If, $\analysis{e}{\tau}$, then by use of the subsumption rule we also have that $\analysis{\gap}{\tau}$:
\[\inference[Subsumption]{\synthesis{\gap}{\dyn} & \tau \sim \dyn}{\analysis{e}{\tau}}\] 
As the dynamic type is consistent with any type: $\dyn \sim \tau$.

Then, to find the \textit{minimum synthesis slice}, we can mimic the Hazel type synthesis rules (see \cref{fig:typing}), replacing uses of type analysis with gaps. Creating a judgement $\synthesisslice{e}{\tau}{\rho}$ meaning: \textit{$e$ that synthesises type $\tau$ under context $\Gamma$ produces minimum synthesis slice $\rho$}.

To demonstrate, the expression slice of a variable $x$ can only be either $x$, requiring the use of $x : \tau$ from the context:
\[
\inference[SVar]{x : \tau \in \Gamma & \tau \neq \dyn}{\synthesisslice{x}{\tau}{[x\mid x:\tau]}}\]
But if $x : \dyn$, then the (empty) slice $[\gap, \emptyset]$ also synthesises $\dyn$, so instead use this. 

For functions, we can recursively find the slice of the function body (which synthesises it's type in the original rules, hence having a minimum synthesis slice) and place inside a function. 
If the assumption $x : \tau_1$ was \textit{required} in synthesising that type, then this name must be present in the expression slice and the context slice no longer requires this assumption to type check the sliced function:
\[\inference[\tiny SFun]{\synthesisslice[\Gamma,x:\tau_1]{e}{\tau_2}{[\varsigma \mid \gamma, x : \tau_1]} }{\synthesisslice{\lambda x:\tau_1.\ e}{\tau_1 \to \tau_2}{[\lambda x : \tau_1.\ \varsigma \mid \gamma]}}\]
Otherwise, if $\gamma$ does not use variable $x$ then this binding may be omitted:
\[\inference[\tiny SFunConst]{\synthesisslice[\Gamma,x:\tau_1]{e}{\tau_2}{[\varsigma \mid \gamma]} & x \not \in \mathrm{dom}(\gamma)}{\synthesisslice{\lambda x:\tau_1.\ e}{\tau_1 \to \tau_2}{[\lambda \gap : \tau_1.\ \varsigma \mid \gamma]}}\]

For function applications we can simply omit the argument, while the slice for the function can be obtained as it synthesises it's type.
\[\inference[\tiny SApp]{\synthesisslice{e_1}{\tau_1}{[\varsigma_1 \mid \gamma_1]}}{\synthesisslice{e_1(e_2)}{\tau}{[\varsigma_1(\gap) \mid \gamma_1]}}\]
The remaining rules are in \cref{fig:SynthesisSlices}.

It is \textit{expected} \textbf{(Proof TODO)} that these rules do indeed always produce a slice for any expression which synthesises a type, and that this slice is minimal:
\begin{conjecture}[Correctness]
\label{conj:SynthesisSliceCorrectness}
If $\synthesis{e}{\tau}$ then:
\begin{itemize}
\item $\synthesisslice{e}{\tau}{\rho}$ where $\rho = [\varsigma \mid \gamma]$ with $\synthesis[\gamma]{\varsigma}{\tau}$.
\item For any $\rho' = [\varsigma' \mid \gamma'] \sqsubseteq [e\mid \Gamma]$ such that $\synthesis[\gamma']{\varsigma'}{\tau}$ then $\rho \sqsubseteq \rho'$.
\end{itemize}
\end{conjecture}

\subsubsection{Extension to Type-Indexed Slices}
This mechanism can be trivially extended to type-indexed slices, where types being synthesised can be replaced by slices directly, i.e. \textit{slice synthesis}. See the appendix \textbf{ref AND FIX}.

The only interesting case is the fact that slices for argument types to functions can now be accessed, so must be represented:

\subsection{Criterion 2: Analysis Slices}\label{sec:AnalysisSlices}
A similar idea can be devised for analysis slices. Essentially, we do the opposite of \textit{criterion 1} and omit sub-terms where \textit{synthesis} was used. The objective is to show \textit{how} a checked type is deconstructed to check inner terms.

The useful notion to represent these are \textit{context slices}. It is the terms immediately \textit{around} the sub-term where the type checking deconstructs a checking type:

For example, when checking this term against $\code{Bool} \to \code{Int}$:
\[(\lambda x. \hole^u)\]
The slice context of the inner hole term $\hole^u$, which is required to be consistent with \code{Int}, would be the following:
\[\hlcmaths[yellow!30]{(\lambda} x\hlcmaths[yellow!30]{.} \hole^u \hlcmaths[yellow!30]{)}\]
Intuitively, this means that the \textit{contextual} reason for $\hole^u$ to be required to be an \code{Int} is that it was within a function (and the context was initially checked against $\code{Bool} \to \code{Int}$ and deconstructed to retrieve the return type $\code{Int}$.

For this criterion we consider only the smallest context that resulted in a term being type checked, for example in the following term:
\[1 : \code{Int} : \dyn : \code{Bool}\]
The context that enforced 1 to be checked against an \code{Int} was only the inner annotation. We refer to this as the \textit{checking context}. The term when inside it's checking context will always synthesis a type.\footnote{If it analysed against a type, then this would have it's own checking context. We merge these contexts.}

\textbf{These definitions below are verbose and it might be better to just leave it intuitive for this, with definition deferred to appendix.}

To represent this, we use a notion of a types \textit{flowing} from other types inside a derivation, i.e. if a type is decomposed, then it's parts \textit{flow} from the complete type. This can be formalised by creating a \textit{type flow} rules \textbf{(Ref to appendix)}.\footnote{This type flow is closely related to mode correctness.} Under this flow, every checked type $\tau'$ in a derivation has an \textit{origin} synthesis rule producing \textit{first} type which $\tau'$ flows from.

\textbf{Maybe give example?}

Then, the \textit{checking context} we want to consider is that of the conclusion of the rule containing the source of the type:
\begin{definition}[Checking Context]
\label{def:CheckingContext}
For a derivation $\synthesis{e}{\tau}$ containing a sub-derivation $\analysis[\Gamma'']{e''}{\tau''}$ and where the \textit{origin} of $\tau''$ is another sub-derivation $\synthesis[\Gamma']{e'}{\tau'}$. Then either:
\begin{itemize}
\item $e''$ is a sub-term of $e'$: the checking context of $e''$ is $\C$ such that $e' = \C\{e''\}$.
\item Otherwise, $\synthesis[\Gamma'']{e''}{\tau''}$ must be a premise in another rule whose conclusion is a synthesis $\synthesis[\Gamma''']{e'''}{\tau'''}$ where $e'$ is a sub-term of $e'''$. The checking context is $\C$ such that $e''' = \C\{e''\}$
\end{itemize}
\end{definition}
It is clear that this must then be the smallest context containing derivations of both the \textit{checked expression} and it's \textit{origin}. This checking context can be defined more intuitively using rules (\textbf{see appendix}) and proven to coincide with the definition above.

An \textit{analysis slice} is a slice of a checked term's \textit{checking context}:
\begin{definition}[Analysis Slice]\label{def:analysisslice}
For a derivation $\synthesis{e}{\tau}$, containing a sub-derivation $\analysis[\Gamma']{e'}{\tau'}$ with checking context $\C$. Then we have that $\synthesis[\Gamma]{\C\{e'\}}{\tau''}$. 

An \textit{analysis slice} of $e'$ is a program context slice $\Cs^\gamma$ such that:
\begin{itemize}
\item $\Cs$ is a slice of $\C$, that is, $\Cs \sqsubseteq \C$.
\item $\Cs\{e\}$ still synthesises a type consistent\footnote{$e'$ within the sliced context might now synthesise a \textit{less precise} type.} with $\tau''$ and contains a sub-derivation checking $e'$ against $\tau'$ in checking context $\Cs$.
\item $\gamma$ contains the required assumptions to type the context with a type consistent with $\tau''$: $\analysis[\gamma]{\Cs\{\gap\}}{\tau''}$.
\end{itemize}
\end{definition}

The \textit{minimum analysis slice} is just the minimum under the precision ordering $\sqsubseteq$. And it must be unique:

\begin{conjecture}[Uniqueness]\label{conj:AnalysisSliceUniqueness}
If $\rho$ and $\rho'$ are \textit{minimum analysis slices} for sub-terms $e'$ of $e$ where $\analysis{e}{\tau}$, then $\rho = \rho'$.
\end{conjecture}

This can again be represented by a judgement read as, \textit{$e$ which type checks against $\tau$ in checking context $\C$ has analysis slice $\p$}:
\[\analysisslice{\C}{e}{\tau}{\p}\]

There are two situations which enforce \textit{checking contexts}, annotations:

\[\inference{\analysis{e}{\tau}}{\analysisslice{\cmark : \tau}{e}{\tau}{[\cmark : \tau \mid \emptyset]}}\]
And applications, for which we need a slice of the application for the \textit{argument} type of the function, which has previously been devised for \textit{type-indexed synthesis slices} (\textbf{ref}):
\[\inference{\synthesissliceindexed{e_1}{\p\{\Cs_1^{g_1}\{i_1 \mid \varsigma_1^{\gamma_1}\} \to s_2 \mid \rho\}} & \analysis{e_2}{\type{i_1}}}{\analysisslice{e_1(\cmark)}{e_2}{\type{i_1}}{[(\Cs_1\{\varsigma_1\})(\cmark) \mid \gamma_1 \cup g_1]}}\]
That is, if $e_1$ synthesises some type $\tau_1 \to \tau_2$ (i.e. $\type{i_1} \to \type{s_2}$), then the slice for $\tau_1$ in the context as a slice $e_1$ is $\Cs_1\{\varsigma_1\}$.

Finally, type analysis rules pass the context down (i.e. sub-terms have the same checking context):
\[\inference{\analysisslice{\C}{\lambda x.\ e}{\tau_1 \to \tau_2}{\p}}{\analysisslice[\Gamma, x:\tau_1]{\C \circ (\lambda x.\ \cmark)}{e}{\tau_2}{\mathrm{ret}(\p)}\circ(\lambda \gap.\ \cmark)}\]
Where $\mathrm{ret}(\p)$ takes the part of the slice that was required in synthesising $\tau_2$, defined in \textbf{appendix}. This direct definition is quite complex; checking against \textit{type-indexed slices} representing the synthesis of $\tau_1 \to \tau_2$ is \textit{much easier}.

This definition does indeed find the minimum analysis slice:
\begin{conjecture}[Correctness]\label{conj:AnalysisSliceCorrectness}
If $\synthesis{e}{\tau}$ with sub-derivation $\analysis{e'}{\tau'}$ in checking context $\C$ then we also have that $\analysisslice{\C}{e'}{\tau'}{\p}$ and:
\begin{itemize}
\item $p$ is an analysis slice for $e'$.
\item For any $\p' = [\Cs' \mid \gamma'] \sqsubseteq [\C\mid \Gamma]$ such that $\p'$ is an analysis slice of $e'$ then $\p \sqsubseteq \p'$.
\end{itemize}
\end{conjecture}

In the case of Hazel, it turns out that this minimum context slice never requires typing assumptions, so $\gamma$ is empty. But, other language features may have more complex behaviour\footnote{Type functions, for example.}. \textbf{VERIFY}

\subsubsection{Extension to Type-Indexed Slices}
As mentioned previously, using \textit{type-indexed synthesis slices} calculated via \textit{criterion 1} as input makes analysis slices much easier to calculate. 

Additionally, the rules in this form end up being more closely tied to the Hazel typing rules and hence easier to formalise.

\textbf{See appendix}

\subsection{Criterion 3: Contribution Slices}
This criterion aims to highlight all regions of code which \textit{contribute} to the given type (either synthesised or analysed). Where \textit{contribute} means that if the sub-term changed it's type, then the overall term would\footnote{No matter which type it was changed to specifically.} also, or would become ill-typed. Importantly, we also consider \textit{contexts} for expression which are analysed, again considering any component terms which having their type changed would result in an error when trying to analyse against the type. For example in the following term:

\[(\lambda f: \code{Int} \to \dyn.\ f(1))(\lambda x : \code{Int}.\ x)\]
The terms which \textit{contribute} to the \textit{second}\footnote{Underlined below from now on.} lambda term checking successfully against $\code{Int} \to \dyn$ is everything \textit{except} the $x$ term, while context slice is just the annotation (and required structural constructs). Highlighting related to synthesis will be a darker shade:
\[\hlcmaths[yellow!30]{(\lambda} f\hlcmaths[yellow!30]{: \code{Int} \to \dyn}.\ f(1)\hlcmaths[yellow!30]{)}\underline{\hlcmaths[yellow!70]{(\lambda x : \code{Int}.}\ x\hlcmaths[yellow!70]{)}}\]

Notice that, in any typing derivation the only sub-terms which \textit{can} have their type changed without causing the rule to no longer apply are those which use type \textit{consistency}.\footnote{Note that this logic does \textit{not} extend to globally inferred languages.} The only such rule is \textit{subsumption}. Further, the only time a term could be changed to \textit{any} type and still remain valid is when it is checked for consistency with the dynamic type $\dyn$.

Therefore, this criterion just \textit{omits all dynamically annotated regions} of the program.\footnote{But does not omit the dynamic annotations themselves.} And, the contextual part of the slices are just \textit{analysis slices}.

\begin{definition}[Contribution Slices]\label{def:ContributionSlice}
For $\synthesis{e}{\tau}$ containing sub-derivation $\analysis[\Gamma']{e'}{\tau'}$ with checking context $\C$.

A \textit{contribution slice} of $e'$ is an \textit{analysis slice} for $e'$ in $\C$ paired with an expression typing slice $\varsigma^\gamma$ such that:
\begin{itemize}
\item $\varsigma$ is a slice of $e'$, that $\varsigma \sqsubseteq e'$.
\item Under restricted typing context $\gamma$, that $\varsigma$ checks against any $\tau'_2$ at least as precise as $\tau'$:\footnote{Essentially, sub-terms that check against $\dyn$ also synthesise $\dyn$. Defined this way to include the case of unannotated lambdas (which do not synthesise).}
\[\forall \tau'_2.\ \tau' \sqsubseteq \tau'_2 \implies \analysis[\gamma]{e'}{\tau'_2}\]
\end{itemize}
A contribution slice for a sub-term $e''$ involved in sub-derivation $\synthesis[\Gamma'']{e''}{\tau''}$ where $e'' \neq e'$ is an expression typing slice $\varsigma''^\gamma''$ which also synthesises $\tau''$ under $\gamma''$, that $\synthesis[\gamma'']{\varsigma''}{\tau''}$. Further, any sub-term of $e''$ which has a contribution slice of the above variety, is replaced inside $\varsigma$ by that corresponding expression typing slice.
\end{definition} 
This is most naturally calculated using \textit{type-indexed slices} exactly replicating the Hazel typing rules:

\textbf{Think more about type indexed context slices. Think how to reconstruct }
\newcommand{\s}{\mathdcal{s}}
\begin{figure}[h]
\[\inference[\tiny SConst]{}{\synthesissliceindexed{c}{b\mid c}} \quad
\inference[\tiny SVar]{x : \tau \in \Gamma}{\synthesissliceindexed{x}{\mathrm{map}{(x^{\{x:\tau\}}, \tau)}}}\]
\[ 
\inference[\tiny SFun]{s_1 = (\gap : \cmark)\circ \mathrm{annot}(\tau_1) &\synthesissliceindexed[\Gamma,x:\tau_1]{e}{s_2}\\ s_2 = i_2 \mid \varsigma^\gamma & x \match{\gamma} p}{\synthesissliceindexed{\lambda x:\tau_1.\ e}{(\lambda \cmark.\ \gap) \circ s_1 \to (\lambda \gap : \gap.\ \cmark) \circ s_2 \mid (\lambda p : \tau_1.\ \varsigma)^{\gamma\backslash x:\tau_1}}}\]
\[\inference[\tiny SApp]{\synthesissliceindexed{e_1}{s_1} & s_1 \funmatch s_2 \to s \\ \analysissliceindexed{e_2}{s_2(\cmark)}{s_2'}}{\synthesissliceindexed{e_1(e_2)}{\text{app}(s_2')}}\]
 
\[\inference[\tiny SEHole]{}{\synthesissliceindexed{\hole^u}{\dyn \mid \hole^u}} \quad \inference[\tiny SNEHole]{\synthesissliceindexed{e}{\p\{i \mid \varsigma^\gamma\}}}{\synthesissliceindexed{\hole[e]^u}{\dyn \mid {\hole[\varsigma]^u}^\gamma}}\]
\[\quad 
\inference[\tiny SAsc]{c = \cmark : \text{annot}(\tau) & \analysissliceindexed{e}{\s}{s}}{\synthesis{e : \tau}{\mathrm{app}(s)}}\]

\[\inference[\tiny AFun]{\s \funmatch \s_1 \to \s_2\\ \analysissliceindexed[\Gamma, x:\type{\s_1}]{e}{\s_2 \circ (\lambda x.\ \cmark)}{s_2}}{\analysissliceindexed{\lambda x.e}{\s}{\s_1\{\lambda \gap. \gap\} \to s_2}}\] 
\[\inference[\tiny ASubsume]{\synthesissliceindexed{e}{s}\\ \type{s} \sim \type{\s}}{\analysissliceindexed{e}{\s}{\bigsqcup\mathrm{static}(\type{\s}, s)}}\]
\caption{Contribution Slices}
\label{fig:ContributionSliceRules}
\end{figure} 

Where $\mathrm{map}(\rho, \tau)$ creates a type-indexed slice of type $\tau$ with the slice context $\cmark$ and expression typing slice $\rho$ tagged on all components of $\tau$. Annot is as defined in criterion 1 extended to type-indexed slices. $x \match_\gamma = p$ matches $p$ with $x$ if $x \in \text{dom}(\gamma)$, otherwise returns $\gap$. $\funmatch$ is extended to slices as follows:
\[\p\{s_1 \to s_2 \mid \rho\} \funmatch \p\circ s_1 \to \p \circ s_2\]
\[\p\{\dyn \mid \rho\} \funmatch \p\{\dyn \mid \rho\} \to \p\{\dyn \mid \rho\}\]
static replaces a (sub)slice term if it is in the position of a $\dyn$ on the input. Reconstructing the term is just taking the join down one level \textbf{PROVE, maybe require as a property in type-indexed slices}. 

\subsection{Join Types}\label{sec:JoinTypesTheory}
The Hazel core calculus is very primitive, only consisting of \textit{base types}, \textit{annotations}, and \textit{functions}. Extensions to gradual types \cite{GradualJoins}, and Hazel \cite{MarkedLocalisation}\footnote{Here as the \textit{meet} of the opposite of my \textit{precision} order.}: \textit{if} expressions, \textit{pattern matching}, \textit{sum types} etc. all require\footnote{Or are easiest formulated with.} \textit{join types}. 

A join of two types $\tau_1 \sqcup \tau_2$ (if one exists) is the least precise (most general) type that more precise than both $\tau_1, \tau_2$: that $\tau_1 \sqsubseteq \tau_1 \sqcup \tau_2$ and $\tau_1 \sqsubseteq \tau_1 \sqcup \tau_2$. Therefore, the join is therefore is consistent with both $\tau_1, \tau_2$. Type consistency can be reformulated in terms of joins: $\tau_1, \tau_2$ are consistent \textit{if and only if} they have a join. This is the \textit{order-theoretic} \textbf{(cite)} join with respect to the precision partial order on types. 
For example, the type of an \textit{if statement} would be the join of the types of it's branches.

These add an additional way to generate a \textit{new type} other than by synthesis or from annotations. 

Hence, type flow needs to be extended to allow these. Equally, slices themselves can be joined if they have common contexts (though there is a decision whether to include both branches of a join).

\section{Cast Slicing Theory}\label{sec:CastSlicingTheory}

Fairly trivial, just treat slices as types and decompose accordingly. The whole reason of indexing by type was to allow this.

The idea of it being a minimal expression typing slice producing the same cast doesn't really work here due to dynamics. Explore the maths of this. Either way, it is a useful construct in practice. Exploring this in more detail, looking at \textit{dynamic program slicing} could be a good future direction.

\textbf{Mention and compare with blame tracking}

\subsection{Indexing Slices by Types}


\subsection{Elaboration}
All casts inserted come from type checking, so can be sliced

\subsection{Dynamics}
Ground type casts will be added, but their addition is purely technical and we can treat their reasoning to just be extracting the relevant portion of the original non-ground type.

\subsection{Cast Dependence}
This in combination with the indexed slices could have some nice mathematical properties.

Though, these would need to retrieve information about parts of slices that were lost when decomposing slices. i.e. when a slice $\tau_1 \to \tau_2$ extracts the argument type $\tau_2$, the slicing criterions lose track of the original lambda binding. A way to reinsert these bindings such that we get a minimal term which \textit{Evaluates to the same cast} e.g. But this will have lots of technicalities with correctly tracking the restricted contexts $\gamma'$ for closures (i.e. functions returning functions which have been applied once). \textbf{TALK ABOUT THIS IN THE FURTHER DIRECTIONS SECTION.}

\section{Proofs}\label{sec:Proofs}
Do if there is time. Prove that analysis slice contexts actually make sense, that they maintain a valid term that is \textit{still} minimal.

\section{Type Slicing Implementation}\label{sec:TypeSlicingImplementation}
\subsection{Type Slice Data-Type}\label{sec:TypeSliceDataType}
Detail initial implementation (just tagging existing types). Compare with final implementation, quantitative numbers for improve could be obtained but would require quite a bit of coding work... Polymorphic Variants :))

Recursive nature of type slices might mean that data structure persistence could allow easy usage while automatically keeping low space. A good point about persistence would be very good, especially if I can show difficulties with the automatic version, since mine explictly maintains incremental and global parts (i.e. does not repeat references), maybe mine would use less pointers?

 Using actual expressions has big overhead and also restrictive (must have valid terms, obviously this is what the theory ensures, but I deviate), so I use ids of terms.
\subsubsection{Code Slices}\label{sec:CodeSlices}
id based. Has ctx used but not actually required as I decide to directly store typslices in context (explain how this differs from the theory).

Mention that full slices as in the theory is very inefficient.
\subsubsection{Integration with Existing Type Data-Type}
Use of `Typ. Explain how this allows it to easily be disabled and saves space (maybe).
\subsubsection{Synthesis \& Analysis Slices}
Incremental/Global. Detail the choice to not annotate many analysis slices.
\subsubsection{Mapping Functions}
\subsubsection{Type Slice Joins}
Show example where multiple branches of join highlighted, i.e. $Int \to \dyn$ and $\dyn \to Bool$.

\subsection{Static Type Checking}\label{sec:TypeChecking}
Detail the statics and Info types. Explain the distinction between Self.re and Mode.re, which links very nicely with the synthesis and analysis slice distinctions.

\subsection{Elaboration}\label{sec:Elaboration}
Make casts use type slices, insertion is mostly the same. Just need to ensure that the slices are preserved (i.e. types are threaded through without being replaced)

\subsection{User Interface}
Click on analysis or synthesis slices from context inspector

\section{Cast Slicing Implementation}\label{sec:CastSlicingImplementation}
\subsection{Cast Transitions}

Cast transition and unboxing logic

\subsection{User Interface}
Mainly talk about the Model-view architecture and passing the cursor into the evaluator view to allow 


\section{EV\_MODE Evaluation Abstraction}
Very complex!!!

\section{Indeterminate Evaluation}\label{sec:IndetEval}
\subsection{Futures Data-Type}\label{sec:Futures}
Lazy lists, often infinite
\subsection{Hole Instantiation}\label{sec:HoleInstantiation}
Small Hole hypothesis, quick check


\subsubsection{Choosing which Hole to Instantiate}
Use EV\_Mode to select next hole to instantiate

\subsubsection{Synthesising Terms for Types}
Difficulties instantiating strings...
Functions with or without annotations?

Seems that there was actually a refine hole editor action that I didn't notice?

\subsubsection{Substituting Holes}\label{sec:HoleSubstitutionImplementation}
Detail that this was an unexpected extra task, and is therefore not exactly the same as hole substitution as detailed in Preparation (i.e. no metavars or contexts annotated on holes, but it is enough for the search procedure to work)

\subsection{Cast Laziness}\label{sec:CastLaziness}
Ref the original cast slicing paper, which is not lazy apparently? Laziness sort of breaks the idea that runtime errors evaluate to cast errors. There can be compound values of the wrong type being cast, but the error will only be found upon accessing parts of the compound type.

Making casts eager is a major change to the actual transitions.

Eager casts also catch `spurious' errors (see Evaluation).


\subsection{User Interface}

\section{Evaluation Stepper}\label{sec:Stepper}
\subsection{Evaluation Contexts}
\subsection{Customisable Hole Instantiation}
\subsection{User Interface}


\section{Search Procedure}\label{sec:SearchProcedure}
\subsection{Detecting Relevant Cast Errors}
i.e. failed cast at head of term
\subsection{Filtering Indeterminate Evaluation}
Done via EV\_MODE similarly to finding which hole to instantiate.

\subsection{Iterative Deepening}\label{sec:IterativeDeepening}
Required after evaluating that infinite loops break the thing
\subsection{User Interface}


