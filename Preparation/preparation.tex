\chapter{Preparation}
\textit{In this chapter I present the technical background knowledge for this project, this is an introduction to the type theory for understanding Hazel's core semantics followed by an overview of Hazel itself. Following this I present my software engineering methodology.}

\section{Background Knowledge}

\subsection{Types}
\subsubsection{Judgements}

\subsubsection{Bidirectional Type Systems}

\subsubsection{Gradual Type Systems}

\subsubsection{Contextual Modal Type Theory}

\subsubsection{System F}

\subsection{The Hazel Language}
\index{Hazel}

\subsubsection{\textbf{Core Hazel}: Formal Semantics}
For reference, the established semantics and type system for Hazel is presented. Derived from Omar et al.\ \cite{HazelLivePaper}. The paper itself goes into deeper depth into the intuition of the rules and the formal properties satisfied by the calculus.
\index{\textbf{Core Hazel}}
\footnote{
\textbf{Come up with and detail some good intuitions of what each type of value \& expression is and what the cast calculus and elaboration does.}}
\subsubsection{Syntax}
\index{\textbf{Core Hazel} syntax}
\par The syntax, in Fig. \ref{fig:syntax}, consists of \textit{types} $\tau$, \textit{external expressions} $e$, and \textit{internal expressions} $d$. Here, $?$ is the \textit{dynamic type}, $\hole[e]$ is a \textit{non-empty hole} containing $e$, and $\scast{\tau_1}{\tau_2}$,  $\scasterror{\tau_1}{\tau_2}$ are casts and cast errors from $\tau_1$ to $\tau_2$ respectively.
The \textit{external language} is a locally inferred \cite{LocalInference} surface syntax for the language, and is statically elaborated to (explicitly typed) \textit{internal expressions}, in a similar way to Harper and Stone's \cite{StandardMLTypeTheory} approach to defining Standard ML as elaboration to an explicitly typed internal langauge, \textit{XML} \cite{CoreXML}.
\begin{figure}[h]
\begin{align*}
\tau &::= b \mid \tau \to \tau \mid\  ?\\
e &::= c \mid x \mid \lambda x : \tau.e \mid \lambda x. e \mid e(e) \mid \hole^u \mid \hole[e]^u \mid e : \tau\\
d &::= c \mid x \mid \lambda x : \tau d \mid d(d) \mid \hole^u_\sigma \mid \hole[d]^u_\sigma \mid d\scast{\tau}{\tau} \mid d\scasterror{\tau}{\tau}
\end{align*}
\caption{Syntax: \textit{types} $\tau$, \textit{external expressions} $e$, \textit{internal expressions} $d$. With $x$ ranging over variables, $u$ over hole names, $\sigma$ over $x \to d$ \textit{internal language} substitutions/environments, $b$ over base types and $c$ over constants.}
\label{fig:syntax}
\end{figure}

\subsubsection{External Language: Type System}
\index{External language type system}
The static semantics in Fig. \ref{fig:typing} of the \textit{external language} is a bidirectionally typed system in the style of Pierce and Turner \cite{LocalInference}, and Dunfield and Krishnaswami \cite{BidirectionalTypes}. There are two typing judgement modes: $\synthesis{e}{\tau}$ which synthesises a type $\tau$, algorithmically thought of as an output, and $\analysis{e}{\tau}$ which analyses against a type $\tau$ as an input.\par 
\begin{figure}[H]
\small
\fbox{$\synthesis{e}{\tau}$}\ \ \ $e$ synthesises type $\tau$ under context $\Gamma$
\tiny
\[\inference[\tiny SConst]{}{\synthesis{c}{b}} \quad
\inference[\tiny SVar]{x : \tau \in \Gamma}{\synthesis{x}{\tau}} \quad 
\inference[\tiny SFun]{\synthesis[\Gamma,x:\tau_1]{e}{\tau_2}}{\synthesis{\lambda x:\tau_1.e}{\tau_1 \to \tau_2}}\]
\[\inference[\tiny SApp]{\synthesis{e_1}{\tau_1} & \tau_1 \funmatch \tau_2 \to \tau \\ \analysis{e_2}{\tau_2}}{\synthesis{e_1(e_2)}{\tau}} \quad 
\inference[\tiny SEHole]{}{\synthesis{\hole^u}{?}} \quad \]
\[\inference[\tiny SNEHole]{\synthesis{e}{\tau}}{\synthesis{\hole[e]^u}{?}}\quad 
\inference[\tiny SAsc]{\analysis{e}{\tau}}{\synthesis{e : \tau}{\tau}}\]

\small
\fbox{$\analysis{e}{\tau}$}\ \ \ $e$ analyses against type $\tau$ under context $\Gamma$

\tiny
\[\inference[\tiny AFun]{\tau \funmatch \tau_1 \to \tau_2\\ \analysis[\Gamma, x:\tau_1]{e}{\tau_2}}{\analysis{\lambda x.e}{\tau}} \quad 
\inference[\tiny ASubsume]{\synthesis{e}{\tau}\\ \tau \sim \tau'}{\analysis{e}{\tau'}}\]

\caption{Bidirectional typing judgements for \textit{external expressions}}
\label{fig:typing}
\end{figure} 
These rules use a type consistency relation, $\sim$ in Fig. \ref{fig:consistency}, with types being consistent if they are equivalent up to the locations of the dynamic type. The type consistency relation is standard in gradual type systems \cite{GradualFunctional, GradualObjects}, and is similar to a subtyping relation but is \textit{not} transitive.

\begin{figure}[H]
\small
\fbox{$\tau_1 \sim \tau_2$}\ \ \ $\tau_1$ is consistent with $\tau_2$
\tiny
\[\inference[\tiny TCDyn1]{}{? \sim \tau} \quad \inference[\tiny TCDyn2]{}{\tau \sim ?} \quad \inference[\tiny TCRfl]{}{\tau \sim \tau} \quad \inference[\tiny TCFun]{\tau_1 \sim \tau_1' & \tau_2 \sim \tau_2'}{\tau_1 \to \tau_2 \sim \tau_1' \to \tau_2'}\]
\caption{Type consistency}
\label{fig:consistency}
\end{figure}
Finally, a (function) type matching relation, $\funmatch$ in Fig. \ref{fig:typematching}, matches the argument and return types from a function type, which for the dynamic type is $? \funmatch ? -> ?$. 
\begin{figure}[h]
\small
\fbox{$\tau \funmatch \tau_1 \to \tau_2$}\ \ \ $\tau$ has arrow type $\tau_1 \to \tau_2$
\tiny
\[\inference[\tiny MADyn]{}{? \funmatch ? \to ?} \quad 
\inference[\tiny MAFun]{}{\tau_1 \to \tau_2 \funmatch \tau_1 \to \tau_2}\]
\caption{Type Matching}
\label{fig:typematching}
\end{figure}


\subsubsection{Elaboration}
Elaboration to the \textit{internal language} is possible for well-typed \textit{external expressions} and consists of cast insertion, maintaining a hole context, and inserting initial identity hole environments. Each of these are used in the internal language type assignment $\typeassignment{e}{\tau}$ and the dynamic semantics. Fig. \ref{fig:elaboration} defines the elaboration judgements and Fig. \ref{fig:typeassignment} defines the internal language type assignment categorical judgement \cite{ModalJudgements}.

\begin{figure}[H]
\small
\fbox{$\elaborationSynthesis{e}{\tau}{d}{\Delta}$}\ \ \ $e$ syntheses type $\tau$ and elaborates to $d$
\tiny 
\[\inference[\tiny ESConst]{}{\elaborationSynthesis{c}{b}{c}{\emptyset}} \quad 
\inference[\tiny ESVar]{x : \tau \in \Gamma}{\elaborationSynthesis{x}{\tau}{x}{\emptyset}}\]
\[\inference[\tiny ESFun]{\elaborationSynthesis[\Gamma,x:\tau_1]{e}{\tau_2}{d}{\Delta}}{\elaborationSynthesis{\lambda x:\tau_1.e}{\tau_1 \to \tau_2}{\lambda x:\tau_1. d}{\Delta}}\]
\[\inference[\tiny ESApp]{\synthesis{e_1}{\tau_1} & \tau_1 \funmatch \tau_2 \to \tau \\ \elaborationAnalysis{e_1}{\tau_2 \to \tau}{d_1}{\tau_1'}{\Delta_1} & \elaborationAnalysis{e_1}{\tau_2}{d_2}{\tau_2'}{\Delta_2}}{\elaborationSynthesis{e_1(e_2)}{\tau}{(d_1\scast{\tau_1'}{\tau_2 \to \tau})(d_2\scast{\tau_2'}{\tau_2})}{\Delta_1 \cup \Delta_2}}\]
\[\inference[\tiny ESEHole]{}{\elaborationSynthesis{\hole^u}{?}{\hole^u_{\mathrm{id}(\Gamma)}}{u :: \hole[] [\Gamma]}}\]
\[\inference[\tiny ESNEHole]{\elaborationSynthesis{e}{\tau}{d}{\Delta}}{\elaborationSynthesis{\hole[e]^u}{?}{\hole[d]^u_{\mathrm{id}(\Gamma)}}{\Delta, u :: \hole[] [\Gamma]}}\]
\[\inference[\tiny ESAsc]{\elaborationAnalysis{e}{\tau}{d}{\tau'}{\Delta}}{\elaborationSynthesis{e:\tau}{\tau}{d\scast{\tau'}{\tau}}{\Delta}}\]
\small
\fbox{$\elaborationAnalysis{e}{\tau}{d}{\tau'}{\Delta}$}\ \ \ $e$ analyses against type $\tau$ and elaborates to $d$ of consistent type $\tau'$
\tiny 
\[\inference[\tiny EAFun]{\tau \funmatch \tau_1 \to \tau_2 \\ \elaborationAnalysis[\Gamma,x:\tau_1]{e}{\tau_2}{d}{\tau_2'}{\Delta}}{\elaborationAnalysis{\lambda x. e}{\tau}{\lambda x:\tau_1. d}{\tau_1 \to \tau_2'}{\Delta}}\]
\[\inference[\tiny EASubsume]{e \neq \hole^u & e \neq \hole[e']^u \\ \elaborationSynthesis{e}{\tau'}{d}{\Delta} & \tau \sim \tau'}{\elaborationAnalysis{e}{\tau}{d}{\tau'}{\Delta}}\]
\[\inference[\tiny EAEHole]{}{\elaborationAnalysis{\hole^u}{\tau}{\hole^u_{\mathrm{id}(\Gamma)}}{\tau}{u::\tau[\Gamma]}}\]
\[\inference[\tiny EANEHole]{\elaborationSynthesis{e}{\tau'}{d}{\Delta}}{\elaborationAnalysis{\hole[e]^u}{\tau}{\hole[d]^u_{\mathrm{id}(\Gamma)}}{\tau}{u::\tau[\Gamma]}}\]

\caption{Elaboration judgements} 
\label{fig:elaboration}
\end{figure}

\begin{figure}[H]
\small
\fbox{$\typeassignment{d}{\tau}$}\ \ \ $d$ is assigned type $\tau$
\tiny
\[\inference[\tiny TACons]{}{\typeassignment{c}{b}}\quad
\inference[\tiny TAVar]{x : \tau \in \Gamma}{\typeassignment{x}{\tau}}\quad
\inference[\tiny TAFun]{\typeassignment[\Delta;\Gamma,x:\tau_1]{d}{\tau_2}}{\typeassignment{\lambda x:\tau_1. d}{\tau_1 \to \tau_2}}\]
\[\inference[\tiny TAApp]{\typeassignment{d_1}{\tau_2 \to \tau} \\ \typeassignment{d_2}{\tau_2}}{\typeassignment{d_1(d_2)}{\tau}} \quad 
\inference[\tiny TAEHole]{u :: \tau[\Gamma'] \in \Delta \\ \typeassignment{\sigma}{\Gamma'}}{\typeassignment{\hole^u_\sigma}{\tau}}\]
\[
\inference[\tiny TANEHole]{\typeassignment{d}{\tau'} \\ u :: \tau[\Gamma' \in \Delta & \typeassignment{\sigma}{\Gamma'}]}{\typeassignment{\hole[d]^u_\sigma}{\tau}}\quad 
\inference[\tiny TACast]{\typeassignment{d}{\tau_1} & \tau_1 \sim \tau_2}{\typeassignment{d\scast{\tau_1}{\tau_2}}{\tau_2}}\]
\[\inference[\tiny TACastError]{\typeassignment{d}{\tau_1} & \tau_1\text{ ground} & \tau_2\text{ ground} & \tau_1 \neq \tau_2}{\typeassignment{d\scasterror{\tau_1}{\tau_2}}{\tau_2}}\]
\caption{Type assignment judgement for \textit{internal expressions}}
\label{fig:typeassignment}
\end{figure}
Where ground types are base types or one-level unrollings of the dynamic type (each being the \textit{least specific} type for each compound type).

\begin{figure}
\tiny
\[id(x_1:\tau_1, \dots, x_n:\tau_n) := [x_1/x_1, \dots, x_n/x_n]\]
\[\typeassignment{\sigma}{\Gamma'} \text{ iff } \mathrm{dom}(\sigma) = \mathrm{dom}(\Gamma')\text{ and for every } x : \tau \in \Gamma' \text{ then:
} \typeassignment{\sigma(x)}{\tau}\]
\caption{Identity substitution and substitution typing}
\label{fig:substitutiontyping}
\end{figure}

\begin{figure}[h]
\tiny
\fbox{$\tau$ ground}\ \ \ $\tau$ is a ground type
\[\inference[\tiny GBase]{}{b\text{ ground}}\quad \inference[\tiny GDynFun]{}{? \to ?\text{ ground}}\]
\caption{Ground types}
\label{fig:groundtypes}
\end{figure}
This elaboration is proven to produce unique internal expressions and hole contexts, and to preserve well-typedness.

\subsubsection{Internal Language: Dynamic Semantics}
In order to support the ability to evaluate expressions around holes and cast errors, Hazel defines multiple syntax-directed classes of final forms in Fig. \ref{fig:finalforms}. \textit{Final forms} are irreducible expressions.
\begin{itemize}
\item Values -- Constants or functions.
\item Boxed values -- Values or boxed values in one of the two cast forms. These must be unboxed (downcast) before reducing.
\item Indeterminate forms -- Irreducible terms containing holes or are casts errors. Substitution of holes may make these reducible.
\item Final -- All final forms.
\end{itemize}

\begin{figure}[H]
\small
\fbox{$\final$}\ \ \ $d$ is final
\tiny
\[\inference[\tiny FBoxedVal]{\boxedval}{\final}\quad
\inference[\tiny FIndex]{\indet}{\final}\]
\small
\fbox{$\val$}\ \ \ $d$ is a value
\tiny
\[\inference[\tiny VConst]{}{\val[c]}\quad
\inference[\tiny VFun]{}{\val[\lambda x:\tau. d]}\]
\small
\fbox{$\boxedval$}\ \ \ $d$ is a boxed value
\tiny
\[\inference[\tiny BVVal]{\val}{\boxedval}\quad
\inference[\tiny BVFunCast]{\tau \to \tau_2 \neq \tau_3 \to \tau_4 & \boxedval}{\boxedval[d\scast{\tau_1 \to \tau_2}{\tau_3 \to \tau_4}]}\]
\[\inference[\tiny BVDynCast]{\boxedval & \ground}{\boxedval[d\scast{\tau}{?}]}\]
\small
\fbox{$\indet$}\ \ \ $d$ is indeterminate
\tiny
\[\inference[\tiny IEHole]{}{\indet[\hole^u_\sigma]}\quad
\inference[\tiny INEHole]{\final}{\indet[{\hole[d]^u_\sigma}]}\quad
\inference[\tiny IAp]{d_1 \neq d_1'\scast{\tau_1 \to \tau_2}{\tau_3 \to \tau_4} \\ \indet[d_1] & \final[d_2]}{\indet[d_1(d_2)]}\]
\[\inference[\tiny ICastGD]{\indet & \ground}{\indet[d\scast{\tau}{?}]}\quad
\inference[\tiny ICastDG]{d \neq d'\scast{\tau'}{?} & \indet & \ground}{\indet[d\scast{?}{\tau}]}\]
\[\inference[\tiny ICastFun]{\tau_1 \to \tau_2 \neq \tau_3 \to \tau_4 & \indet}{\indet[d\scast{\tau_1 \to \tau_2}{\tau_3 \to \tau_4}]}\quad
\inference[\tiny ICastError]{\final & \ground[\tau_1] \\ \ground[\tau_2] & \tau_1 \neq \tau_2}{\indet[d\scasterror{\tau_1}{\tau_2}]}\]

\caption{Final forms}
\label{fig:finalforms}
\end{figure}


The small-step contextual dynamics \cite{PracticalFoundations} is defined on the internal expressions. The general idea is to consider two classes of casts: injections -- casts from a ground type to the dynamic types, and projections -- casts from the dynamic type to a ground type. These two classes of casts can be eliminated upon meeting if the ground types are equal or to a cast error if not. Function casts are dealt with by separating into two casts on the argument and return value. Finally, compound types can be cast to their least specific ground type specified by the ground matching relation in Fig. \ref{fig:groundmatch}.

\begin{figure}[H]
\small
\fbox{$\tau \groundmatch \tau'$}\ \ \ $\tau$ matches ground type $\tau'$
\tiny
\[\inference[\tiny]{\tau_1 \to \tau_2 \neq ? \to ?}{\tau_1 \to \tau_2 \groundmatch ? \to ?}\]
\caption{Ground type matching}
\label{fig:groundmatch}
\end{figure}

\begin{figure}[H]
\small
\fbox{$d \longrightarrow d'$}\ \ \ $d$ takes and instruction transition to $d'$
\tiny
\[\inference[\tiny ITFun]{}{(\lambda x:\tau. d_1)(d_2) \longrightarrow [d_2/x]d_1}\quad
\inference[\tiny ITCastId]{}{d\scast{\tau}{\tau} \longrightarrow d}\]
\[\inference[\tiny ITAppCast]{\tau_1 \to \tau_2 \neq \tau_1' \to \tau_2'}{d_1\scast{\tau_1 \to \tau_2}{\tau_1' \to \tau_2'}(d) \longrightarrow (d_1(d_2\scast{\tau_1'}{\tau_1}))\scast{\tau_2}{\tau_2'}}\]
\[\inference[\tiny ITCast]{\ground}{d\scastcast{\tau}{?}{\tau} \longrightarrow d} \quad
\inference[\tiny ITCastError]{\tau_1 \neq \tau_2 \\ \ground[\tau_1] & \ground[\tau_2]}{d\scastcast{\tau_1}{?}{\tau_2}\longrightarrow d\scasterror{\tau_1}{?}{\tau_2}}\]
\[\inference[\tiny ITGround]{\tau \groundmatch \tau'}{d\scast{\tau}{?} \longrightarrow d\scastcast{\tau}{\tau'}{?}}\quad
\inference[\tiny ITExpand]{\tau \groundmatch \tau'}{d\scast{?}{\tau} \longrightarrow d\scastcast{?}{\tau'}{\tau}}\]
\caption{Instruction transitions}
\label{fig:instructions}
\end{figure}

Variable substitution, used in ITFun, is capture avoiding and also substitutions are recorded in each hole's environment $\sigma$ by (over)writing $[d/x]\sigma$.\footnote{Create a figure for this, maybe in appendix.}

\begin{figure}[H]
Context syntax:
\[E ::= \circ \mid E(d) \mid d(E) \mid \hole[E]^u_\sigma \mid E\scast{\tau}{\tau} \mid E\scasterror{\tau}{\tau}\]
\small
\fbox{$d = E[d]$}\ \ \ $d$ is the context $E$ filled with $d'$ in place of $\circ$
\tiny
\[\inference[\tiny ECOuter]{}{d = \circ[d]} \quad
\inference[\tiny ECApp1]{d_1 = E[d_1']}{d_1(d_2) = E(d_2)[d_1]}\quad
\inference[\tiny ECApp2]{d_2 = E[d_2]}{d_1(d_2) = d_1(E)[d_2']}\]
\[\inference[\tiny ECNEHole]{d=E[d']}{\hole[d]^u_\sigma = \hole[E]^u_\sigma [d']}\quad
\inference[\tiny ECCast]{d=E[d']}{d\scast{\tau_1}{\tau_2} = E\scast{\tau_1}{\tau_2}[d']}\]
\[\inference[\tiny ECCastError]{d=E[d']}{d\scasterror{\tau_1}{\tau_2} = E\scasterror{\tau_1}{\tau_2}[d']}\]
\small
\fbox{$d \mapsto d'$}\ \ \ $d$ steps to $d'$
\tiny
\[\inference[\tiny Step]{d_1=E[d_2] & d_2 \longrightarrow d_2' & d_1' = E[d_2']}{d_1 \mapsto d_1'}\]
\caption{Contextual dynamics of the internal language}
\label{fig:dynamics}
\end{figure}

The instruction transitions, Fig. \ref{fig:instructions}, and evaluation context, Fig. \ref{fig:dynamics}, specify a non-deterministic evaluation order, which is a more suitable choice for supporting dynamic hole instantiation, as will be used by the search procedure.\footnote{\textbf{Define Substitution in appendices?}}\par 
Casts are associative, so bracketing is omitted.

\subsubsection{Hole Substitutions}
Hole substitutions will be extensively used by the search procedure, so their semantics is given here. These correspond to contextual substitutions for meta-variables in closures in contextual modal type theory \cite{CMTT}. Intuitively, these are delayed substitutions with the holes being closures with environment $\sigma$ and meta-variable/hole name $u$. A hole $\hole^u_\sigma$ being substituted with $d$ (written $\contextualsub \hole_\sigma^u$) is $d$ with it's environment assignment $\sigma$ applied, after first substituting the hole ($\contextualsub \sigma$) in $\sigma$.\par Hole/contextual substitutions are mostly similar variable substitutions, except the substituted values may refer to bound variables without restriction. See Fig. \ref{fig:holesubstitution} for full definition. 
\begin{figure}[H]
\small
\fbox{$\contextualsub d' = d''$}\ \ \ $d''$ is $d'$ with each hole $u$ substituted with $d$ in the respective hole's environment $\sigma$.
\begin{align*}
&\contextualsub c &&= c\\
&\contextualsub x &&= x\\
&\contextualsub \lambda x:\tau.d' &&= \lambda x:\tau. \contextualsub d'\\
&\contextualsub d_1(d_2) &&= (\contextualsub d_1)(\contextualsub d_2)\\
&\contextualsub \hole^u_\sigma &&= [\contextualsub  \sigma]d\\
&\contextualsub \hole^v_\sigma &&= \hole^b_{\contextualsub \sigma} && \text{if $u \neq v$}\\
&\contextualsub \hole[d']^u_\sigma &&= [\contextualsub  \sigma]d\\
&\contextualsub \hole[d']^v_\sigma &&= \hole[\contextualsub d']^b_{\contextualsub \sigma} && \text{if $u \neq v$}\\
&\contextualsub d'\scast{\tau}{\tau'} &&= (\contextualsub d')\scast{\tau}{\tau'}\\
&\contextualsub d'\scasterror{\tau}{\tau'} &&= (\contextualsub d')\scasterror{\tau}{\tau'}\\
\end{align*}

\fbox{$\contextualsub \sigma = \sigma'$}\ \ \ $\sigma'$ is $\sigma$ with each hole $u$ in $\sigma$ substituted with $d$ in the respective hole's environment.
\begin{align*}
\contextualsub \cdot &= \cdot\\ 
\contextualsub \sigma, d'/x &= \contextualsub \sigma, (\contextualsub d')/x
\end{align*}
\caption{Hole substitution}
\label{fig:holesubstitution}
\end{figure} 
\textbf{Think in detail about CMTT and hole refinement. It is very relevant to the search procedure. In particular think about \texttt{box} and \texttt{letbox} in CMTT.} 

\subsection{The Hazel Implementation}
\subsubsection{Additional Features}
\begin{itemize}
\item 
\end{itemize}
\subsubsection{Code Structure}

\subsubsection{Terms}

\subsubsection{Monadic Evaluation}


\section{Starting Point}
\subsubsection{Concepts}
The foundations of most concepts in understanding Hazel from Part IB Semantics of Programming (and Part II Types later). The concept of gradual typing briefly appeared in Part IB Concepts of Programming Languages, but was not formalised. Dynamic typing, gradual typing, holes, and contextual modal type theory were not covered in Part IB, so were partially researched leading up to the project, then researched further in greater depth during the early stages. Similarly, Part IB Artificial Integlligence provided some context for search procedures. Primarily, the OCaml search procedure for ill-typed witnesses Seidel et al. \cite{SearchProc} and the Hazel core language \cite{HazelLivePaper} were researched over the preceding summer.

\subsubsection{Tools and Source Code}
My only experience in OCaml was from the Part IA Foundations of Computer Science course. The Hazel source code had not been inspected in any detail until after starting the project.

\section{Requirement Analysis}

\section{Software Engineering Methodology}

\section{Legality}
