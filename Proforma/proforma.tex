\chapter*{Proforma}
\begin{tabularx}{\linewidth}{lX}
Candidate Number: & \textbf{\candidatenumber}\\
College: & \textbf{\college}\\
Project Title: & \textbf{Type Error Debugging in Hazel}\\
Examination: & \textbf{\tripos\ -- \submissiondeadline}\\
Word Count: & \textbf{\wordcount}\footnote{\textit{Words in text calculated by \texttt{texcount}. Including: tables and footnotes. Excluding: the front matter,  bibliography, and appendices}}\\
Code Line Count: & \textbf{10380} \footnote{\textit{Git diff, between \texttt{Evaluation} branch and \code{dev} branch. Includes \texttt{.ml}, \texttt{.re}, dune files, and \texttt{.sh} files. Excludes \texttt{/evaluaton/data/*}}}\\
Project Originator: & \textbf{\projectoriginator}\\
Supervisors: & \textbf{\supervisors}
\end{tabularx}

\section*{Original Aims of the Project}
This project seeks to explore ways to improve \textit{type error} debugging. Traces to dynamic type errors can provide better intuition to \textit{why} static type errors were found. Additionally, dynamic errors do not often point directly back to source code, or do so only incompletely. This project aimed to create a source code highlighting system for dynamic errors and an automated search procedure for dynamic errors. The Hazel language was chosen, allowing the project to explore both static and dynamic errors and their interaction. Only a subset of Hazel was expected to be supported, but enough to demonstrate the methods' promise.

\section*{Work Completed}
The project successfully supports \textit{almost all} of Hazel. Hazel is an intensely active research project so significant work went into staying up to date and producing a corpus of Hazel programs. Four searching methods were implemented, the best failing to find existing dynamic errors for only 2\% of the corpus. Dynamic error highlighting had no prescribed basis to build upon. Therefore, a very significant part was devising \textit{novel} formalisations for type-directed highlighting systems which propagate information throughout evaluation; further extended to explain \textit{static} errors also. Both error highlighting implementations were found to be concise and demonstrated to be effective.
\section*{Special Difficulties}
None