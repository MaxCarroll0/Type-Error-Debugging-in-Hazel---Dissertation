\chapter{Introduction}
\label{chap:Introduction}
Software bugs are an inherent part of programming, often leading to unexpected behaviour and system failures. Debugging these errors is a \textit{time-consuming process} taking between 20-60\% of active work time \cite{DebugTimeSelfReport}, with programmers spending a \textit{highly skewed} proportion of their time identifying and resolving a small proportion of \textit{difficult} bugs \cite{DebugSkew}.

Type systems aim to alleviate some of this burden by classifying expressions and operations that are allowed to work on them. This may be done \textit{statically} at compile time or \textit{dynamically} during runtime. The expressions not conforming to the type system manifest themselves as \textit{type errors}.

In static typing, blame for type errors are typically localised to a \textit{single} location in the code. However, this localisation may be misleading, as the actual cause of the error might be rooted in a broader context, for example in OCaml 65\% of type errors related to \textit{multiple} locations \cite{StudentTypeErrorFixes}. This is a particularly prevalent issue in \textit{type inferred} languages.

In dynamic typing, type errors are often missed as they only appear during runtime with specific inputs.
Additionally, they don't generally specify any source code context which caused them. Instead, a dynamic type error is accompanied by an \textit{evaluation trace}, which can be \textit{more intuitive} \cite{TraceVisualisation} by demonstrating concretely why values are \textit{not} consistent with their expected type as required by a \textit{runtime cast}.

This project seeks to improve user understanding of type errors by localising static type errors \textit{more completely}, and \textit{combining} the benefits of static and dynamic type errors.

I consider three research problems and implement three features to solve them in the Hazel language \cite{Hazel}:\footnote{The answers may differ greatly for other languages. Hazel provides a good balance between complexity and usefulness.}
\begin{enumerate}
\item Can we statically highlight code which explains \textit{static type errors} more \textit{completely}, including all code that \textit{contributes} to the error? 

This would alleviate the issue of static errors being \textit{incorrectly localised}, and help give a \textit{greater context} to static type errors.

\textbf{Solution:} I devise a \textit{novel} method of \textbf{type slicing} including formal mathematical foundations built upon the formal \textit{Hazel calculus} \cite{HazelLivePaper}. Additionally, it generalises to highlight all code relevant to typing \textit{any} expressions (not just erroneous expressions).

\item Can we dynamically highlight source code which contributes to a \textit{dynamic type error}?

This would provide missing source code context to understand how types involved in a dynamic type error originate from the source code.

\textbf{Solution:} I devise a \textit{novel} method of \textbf{cast slicing}, also with formal mathematical foundations. Additionally, it generalises to highlight source code relevant to requiring any specific \textit{runtime casts}.

\item Can we provide dynamic \textit{evaluation traces} to explain \textit{static type errors}?

This would provide an \textit{intuitive} concrete explanation for static type errors.

\textbf{Solution:} I implement a \textbf{type error witness search procedure}, which discovers inputs (witnesses) to expressions which cause a \textit{dynamic type error}. This is based on research by Seidel et al. \cite{SearchProc} which devised a similar procedure for a subset of OCaml.
\end{enumerate}

Hazel \cite{Hazel} is a functional locally inferred and gradually typed research language allowing the writing of \textit{incomplete programs} under active development at the University of Michigan. 

\textbf{INTRODUCE HAZEL \& IT'S VISION HERE WITH IT'S BASIC UNUSUAL FEATURES \& Show why it is a good choice to answer the research questions for}.

(i.e. a notebook environment, easy teaching language for students, teaching language to understand complex type systems... All are situations where better error explanations are useful (\textbf{cite that students struggle more with type errors}))

These three features \textit{work well together} in Hazel to allow \textit{both} static and dynamic type errors to be explained to a \textit{greater extent} than in any existing languages. Arguably, these explanations are \textit{intuitive}, and should help reduce debugging times and aid in students understanding type systems.\footnote{Proving this would require a user study.}

For example, here is a basic walk-through for how a static type error could be diagnosed using the three features. 

\textbf{MAP function? show a static error version}
\textbf{Reference to detailed analysis in the evaluation section.}

\begin{figure}

\caption{A Static Type Error}
\label{fig:ErrorExample}
\end{figure}

\section{Related Work}
\label{sec:RelatedWork}
There has been extensive research into attempting to understand \textit{what} is needed \cite{DebugNeeds}, \textit{how} developers fix bugs \cite{HowFixBugs}, and a plethora of compiler \textit{improvements} and \textit{tools} \textbf{add citations here, primarily functional language tools}. This project builds upon this body of research in new ways focusing on the Hazel language, which is itself a research project being taken in various directions (\textbf{cite the directions}) but generally as a \textit{teaching language} (\textbf{cite the explainthis paper}) for students. 

To my knowledge the ideas of \textit{type slicing} and \textit{cast slicing} are novel. However, they do output \textit{program slices} which were originally explored by Weiser \cite{ProgSlice}, though my definition of program slices matches more with functional program slices \cite{FunctionalProgExplain}.  The properties I explore are more similar to \textit{dynamic program slicing} \cite{DynProgSlice} but with type characteristics rather than evaluation characteristics, in a sense similar \textit{type error slicing} \cite{ErrSlice}.

The \textit{type witness search procedure} is	 based upon Seidel et al. \cite{SearchProc}, though there are significant differences in workings due to my use of Hazel and various extensions as compared to the subset of OCaml used by Seidel et al.

\section{Dissertation Outline}
\label{sec:Outline}
\Cref{chap:Preparation} introduces the \textit{type theories} (\cref{sec:TypeSystems}) underpinning the \textit{Hazel core calculus} (\cref{sec:CoreHazel}), followed by basic background on the \textit{Hazel implementation} (\cref{sec:HazelImplementation}). Additionally, methods of non-deterministic programming are introduced (\cref{sec:Nondeterminism}).

\Cref{sec:TypeSlicingTheory} and \cref{sec:CastSlicingTheory} formalise and prove\footnote{TODO} the ideas of \textit{type slices} and \textit{cast slices} in the Hazel core calculus. 
 
 An implementation (\cref{sec:TypeSlicingImplementation}-\ref{sec:CastSlicingImplementation}) of these has been created covering \textit{most}\footnote{Except for type substitution.} of the Hazel language, including a user interface. 
 
 A \textit{type witness search procedure} was successfully implemented. This involved creating a \textit{hole instantiation} and a simplified \textit{hole substitution} method (\cref{sec:HoleInstantiation}); there is currently no full hole substitution feature in Hazel despite it's presence in the core calculus (\cref{sec:HoleSubstitution}). Additionally, a customisable instantiation method was implemented (\cref{sec:Stepper}) controllable via a UI.\footnote{TODO}

The slicing features and search procedure met all goals, \textbf{list eval goals briefly}, showing \textit{effectiveness} (\cref{sec:EffectivenessAnalysis}) and being reasonably \textit{performant} (\cref{sec:PerformanceAnalysis}) over a corpus of \textit{well-typed} and \textit{ill-typed} Hazel programs respectively (\cref{sec:CorpusCollection}). Further, considered deviations from the slicing theories and strengths and weaknesses of the search procedure were evaluated in detail (\cref{sec:CriticalAnalysis}). A cognitive walk-through is performed in \cref{sec:CognitiveWalkthrough} considering the use of all the features in debugging the static type error presented in the introduction in \cref{fig:StaticError}.

Finally, further directions and improvements have been presented along with discussion on the applicability of these features, slicing in particular, to real world debugging situations in \cref{chap:Conclusions}.